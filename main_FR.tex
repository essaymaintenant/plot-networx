%
\documentclass[fleqn,treatise,allpages]{ISTE_Science}
\usepackage[english,french]{babel}
%\documentclass[fleqn,french,treatise,allpages]{ISTE_Science}
%\raggedbottom
%\CropMarksOff
%\nofiles
%\sloppy

\setlength{\textwidth}{120mm}  % 120mm = 341pt
\setlength{\textheight}{184mm} % 183mm = 521pt (43.4 lignes * 12pt)

%--- Dimensions horizontales

\setlength{\hoffset}{-1in}            % ‡ ajuster par dvips selon imprimante
\setlength{\oddsidemargin}{45mm}      % pour que la marge de gauche
\setlength{\evensidemargin}{45mm}     % soit exactement de 45mm
  % Largeur de la marge de droite pour papier:
  %  - A4     : largeur papier = 210 mm => 210 - (120+45)  = 45  mm
  %  - Letter : 8.5in = 8.5*25.4mm = 215.9mm => 215.9-165  = 50.9mm

%--- Dimensions verticales

\setlength{\voffset}{-1in}       % …limination de la marge par dÈfaut de 1 pouce

%\setlength{\topmargin}{29.5mm}     % distance bord sup feuille au haut de l'en-tÍte
%\settoheight{\headheight}{2.5mm} % hauteur de l'en-tÍte
%\setlength{\headsep}{7.4mm}      % marge entre corps du texte et en-tÍte
%                                 % => 57mm = 47mm + 2.5mm + 7.5mm = marge au dessus du
%                                 % texte d'une feuille A4 non massicotÈe

\setlength{\topmargin}{47mm}     % distance bord sup feuille au haut de l'en-tête
\settoheight{\headheight}{2.5mm} % hauteur de l'en-tête
\setlength{\headsep}{7.5mm}      % marge entre corps du texte et en-tête


\makeatletter
\let\bbl@nonfrenchlistspacing\relax
\makeatother
\usepackage{marginnote,array}
\usepackage{lscape}
\usepackage{qtree}
\usepackage{fancybox}
\usepackage{makeidx}
\usepackage[none]{hyphenat}


\def\latdeg{$^{\circ}$}
\def\latmin{$^\prime$}



%\def\x{{\mathbf x}}
%\def\L{{\cal L}}

%\def\test{\mathcal{T}}
\def\wp{\mathbf{W}}
\def\wpf{W}
\def\versus{\leftrightarrow}


%\wavelettransform
\def\wavelettransform{\mathcal{T_W}}
\def\googletransform{\mathcal{T_X}}
\def\alextransform{\mathcal{T_A}}
\def\alexnet{\mathcal{A}}
\def\wavenet{\mathcal{W}}
\def\googlenet{\mathcal{X}}
\def\deepcoeff{\mathfrak{C}}
\def\googlecoeff{\mathfrak{C}^\googlenet}
\def\alexcoeff{\mathfrak{C}^\alexnet}
\def\wavenetcoeff{\mathfrak{C}^\wavenet}
\def\swpcoefc{\mathfrak{C}}


\def\moy{{\bf \mu}}
\def\cova{{\bf \Sigma}}
\def\ncf{S}
\def\aa{\alpha}
\def\bb{\beta}
\def\cumdis{{\lambda}}
\def\voisins{\mathbb{O}}
%\def\setminus{\mathbb{O}}


\def\pdf{\mathfrak{f}}
\def\keypoint{\mathfrak{x}}
\def\bincenter{\keypoint}


\def\Nc{N}
\def\klmap{{\bf K}}
\def\Map{\mathcal{M}}
\def\K{\mathcal{K}}
\def\H{\mathcal{H}}



\def\ppp{p}
\def\qqq{q}
\def\param{\mathbf{\theta}}

\def\vector{\mathfrak{V}}
\def\imgo{\mathcal{I}}


\def\cum{\kappa}
\def\cscanmap{\mathcal{M}}
\def\separ{/}
\def\scaleparam{\rho_1}
\def\shapeparam{\rho_2}
\def\dif{\mathrm{d}}
\def\gammaln{\Lambda}
\def\trigamma{\psi^{(2)}}
\def\gammaa{\mathfrak{e}}



\def\wpixel{{{\mathcal V}\pixel}}
\def\dkl{\mathfrak D}
\def\xpixel{u}
\def\ypixel{v}
\def\pixel{{[\xpixel, \ypixel]}}
\def\imgopixel{\mathbf{a}}



\definecolor{mygreen}{rgb}{0.25,0.50,0.50}
%\definecolor{mg}{rgb}{0.25,0.50,0.50}
\newcommand{\noir}{\textcolor{black}}
\newcommand{\bleu}{\textcolor{blue}}


\newcommand\red[1]{\textcolor{red}{#1}}
%\newcommand\removed[1]{\textcolor{gray}{\st{#1}}} % with package soul
\newcommand\blue[1]{\textcolor{blue}{#1}}
\newcommand\green[1]{\textcolor{green}{#1}}
\newcommand\white[1]{\textcolor{white}{#1}}
\newcommand\magenta[1]{\textcolor{magenta}{#1}}



\newcommand\myred[1]{\textcolor{red}{#1}}
\newcommand\removed[1]{\textcolor{gray}{\st{#1}}} % with package soul
\newcommand\myblue[1]{\textcolor{blue}{#1}}
%\newcommand\mygreen[1]{\textcolor{green}{#1}}
\newcommand{\mygreen}{\textcolor{mygreen}}
\newcommand\mymagenta[1]{\textcolor{magenta}{#1}}
\newcommand\mybrown[1]{\textcolor{brown}{#1}}
\newcommand\mycyan[1]{\textcolor{cyan}{#1}}


%\newcommand\mymagenta[1]{\textcolor{magenta}{#1}}
\newcommand\myyellow[1]{\textcolor{yellow}{#1}}
%\newcommand\mymagenta[1]{\textcolor{magenta}{#1}}

\definecolor{blueuds}{rgb}{0.04,0.23,0.44}
\newcommand{\blueuds}{\textcolor{blue_uds}}
%%%
\definecolor{bluepacy}{rgb}{0.2,0.5,0.8}
\newcommand{\bluepacy}{\textcolor{blue_pacy}}
%%%
\definecolor{mygreenlight}{rgb}{0.6,0.8,0}
\newcommand{\mygreenlight}{\textcolor{mygreenlight}}
%%%
\definecolor{mygreenlight2}{rgb}{0.74,0.74,0}
\newcommand{\mygreenmoonlight}{\textcolor{mygreenlight2}}

\def\imgorig{\imageseries}
\def\imgorig{\mathcal{I}}
%\def\img{\mathcal{T}}
\def\mspectrum{{\overline{\gamma}}}
\def\wproj{{\cal W}}
\def\wnode{\mathbf{\theta}}
\def\swpcoef{{\bf {C} }}
\def\oacc{\{}
\def\facc{\}}
\def\img{\mathcal{T}}
\def\rrr{r}
\def\variance{\textrm{var}}
\def\fperm{F}
\def\m{\epsilon}
\def\swpvarvect{{\bf {C} }}
\def\armafuncden{\Phi}
\def\armafuncnum{\Theta}

\newcommand{\argmin}{\arg\!\min}
\newcommand{\argmax}{\arg\!\max}
\newcommand{\dist}{dist}
\newcommand{\Max}{max}
\newcommand{\Min}{min}
\newcommand{\pw}{PW}
\newcommand{\nlm}{NLM}
\newcommand{\CM}{CM}
\newcommand{\LR}{LR}
\newcommand{\x}{x}
\newcommand{\y}{y}


%FROM CHAPTER 14   //// SUPPRIMER DU MAIN14 PAR ABDOU
%\usepackage{subcaption}  % conflit avec subfigure
%\captionsetup{compatibility=false}
%\usepackage{graphicx,booktabs}
%\usepackage{capt-of}
%\usepackage{pdflscape}
%\usepackage{afterpage}
%\usepackage{mwe}
\usepackage{amsmath}
%\usepackage{pdflscape}


%\usepackage{pgf}
\usepackage{layouts}
%\usepackage{printlen}

\usepackage{slashbox}

\usepackage{colortbl}


%\usepackage{subcaption}
%\usepackage{subfig}
%\usepackage{subfigure}
%
%\usepackage{subcaption}

%%% ------------------------------------ 
\usepackage{mathabx}
%%% ------------------------------------ 
\usepackage{IEEEtrantools}

\def\theequationdis{{\normalfont \normalcolor [\theequation]}}% (1)
\def\theIEEEsubequationdis{{\normalfont \normalcolor [\theIEEEsubequation]}}% (1a)
\def\@eqnnum{\theequationdis}




\makeatletter

\setcounter{MaxMatrixCols}{30}
\usepackage{amsmath, bbm, bm}
\usepackage{array,longtable,booktabs}
\usepackage{multirow}
% package chapter 2

%%% ------------------------------------ 
\usepackage{pgf}
\usepackage{pgfplots}
\pgfplotsset{compat=newest}
\usepgfplotslibrary{groupplots}
\usepgfplotslibrary{dateplot}
\usepgfplotslibrary{external}
%%% ------------------------------------ 

\usepackage{tikz}
\usetikzlibrary{patterns}
\usetikzlibrary{3d}
%\tikzexternalize[prefix=tikz_figures/]
\usepackage[nomessages]{fp}% http://ctan.org/pkg/fp
\def\checkmark{\tikz\fill[scale=0.4](0,.35) -- (.25,0) -- (1,.7) -- (.25,.15) -- cycle;}
\usepackage[ruled,vlined,algochapter]{algorithm2e} % algorithm
\SetAlgoCaptionSeparator{.--}
\usetikzlibrary{matrix,shapes.multipart}
\newcommand{\thealgorithm}{\thechapter.\arabic{algorithm}.}

% package chapter 15
\usepackage{booktabs}
\usepackage{multirow}
\usepackage{paralist}
%\usepackage{hyperref}
\usepackage{multirow,multicol}
\usepackage{enumerate}
%
%
%
%
%
\newcommand{\Le}{\boldsymbol{\ell}}
\newcommand{\bld}[1]{{\mathbf #1}}
\newcommand{\p}{\, \textup{p} \,}
\newcommand{\Io}[1]{\mathrm{I_0}\left( #1 \right)}
\newcommand{\Iu}[1]{\mathrm{I_1}\left( #1 \right)}
\newcommand{\Id}[1]{\mathrm{I_2}\left( #1 \right)}
\newcommand{\Ju}[1]{\mathrm{J_1}\left( #1 \right)}
\newcommand{\Jd}[1]{\mathrm{J_2}\left( #1 \right)}
\newcommand{\II}[2]{\mathrm{I_{#1}}\left( #2 \right)}
\newcommand{\I}{\mathrm{I}}
\newcommand{\X}{\boldsymbol{\mathrm{X}}}
\newcommand{\Y}{\boldsymbol{\mathrm{Y}}}
\newcommand{\D}{\boldsymbol{\mathrm{D}}}
\newcommand{\Q}{\boldsymbol{\mathrm{Q}}}
\newcommand{\N}{\mathcal{N}}
\newcommand{\Ri}{\mathcal{R}}
\newcommand{\f}{\, \textup{f}}
\def\myvar{0.17}
\def\myvarA{0.3}
\def\myvarB{0.27}
\newcommand{\dataset}[1]{\noindent\textbf{#1:}}




\DeclareMathOperator{\diag}{Diag}
\DeclareMathOperator{\sgn}{sgn}
\DeclareMathOperator{\card}{card}





% BEGIN notations Chapter 8
\usepackage{xcolor,multirow}
\def\vectorimageseries{\boldsymbol{\mathcal{I}}} % Image Time Series
\def\labelseries{\mathcal{M}}
\def\quadtrees{\mathcal{S}}
\def\xx{\boldsymbol{x}}
\def\xd{\boldsymbol{X}}
\def\cc{c}
\definecolor{color_legend_water}{RGB}{128,255,255}
\definecolor{color_legend_urban}{RGB}{255,255,128}
\definecolor{color_legend_vegetation}{RGB}{0,128,0}
\definecolor{color_legend_bare}{RGB}{192,192,192}
% END notations Chapter 8


% Our notations (do not modify)
\def\imageseries{\mathcal{I}} % Image Time Series
\def\Nset{\mathbb{N}}
\def\Zset{\mathbb{Z}}
\def\Rset{\mathbb{R}}
\def\Cset{\mathbb{C}}

\def\Expect{\mathbb{E}}
\def\MeanScalar{\mu}
\def\MeanVect{\bm{\mu}}
\def\CovMatPhys{\mathbf{C}}
\def\CovMatStats{\mathbf{\Sigma}}

\newsavebox{\fminibox} \newlength{\fminilength} \newenvironment{fminipage}[1][\linewidth]{%
 \setlength{\fminilength}{#1 - 2\fboxsep - 2\fboxrule}
 \begin{lrbox}{\fminibox}
  \begin{minipage}{\fminilength}}{%
  \end{minipage}
 \end{lrbox}
 \noindent
 \fbox{\usebox{\fminibox}}}

 % command and notations for chapter 2
\newcommand\supinf{\underset{\mathrm{H}_{0}}{\overset{\mathrm{H}_{1}}{\gtrless}}} % Great Less for Detectors


\title[D\'etection de changements et analyse de s\'eries chronologiques d'images 1]{D\'etection de changements et analyse de s\'eries chronologiques d'images 1}



%\author{List of authors}
%\author{%
%Roger \Name{Rousseau} (class, styles, and tools design)\\[2pt]
%Christian \Name{Scheen} (English documentation)}

%\date{%
%Version~\PackageVersion{}, \filedate{}}



\usepackage[sectionbib]{natbib}
\usepackage{chapterbib}
\usepackage{subfiles} % Best loaded last in the preamble
\renewcommand\bibsection{\section{\bibname}}
\setlength{\bibsep}{5pt}
\renewcommand{\bibfont}{\fontsize{10}{12}\selectfont}
\setcitestyle{citesep={;}}
\setcitestyle{aysep={}}
\setcitestyle{yysep={,~}}
\let\cite=\citep

\makeatletter
\def\@makefnmark{\raisebox{0.65ex}{\FonteAppelNote\@thefnmark}}
\renewcommand{\@makefntext}[1]{%
   % Place une footnote en bas de page avec le numéro \thefootnote
   % \@prevpage sert à détecter les notes de bas de page à cheval sur 2 pages
   \FonteNoteBasPage %NEW
   \noindent\@Myfnmark#1 %
}%
\makeatother


\newcommand{\mpapp}{D\'etection de changements et analyse de s\'eries chronologiques d'images 1}

\newcommand{\myauthor}{Abdourrahmane M. {\sc Atto}, Francesca {\sc Bovolo} et Lorenzo {\sc Bruzzone}}


%\makeindex


\begin{document}

\makeatletter
\lefthyphenmin=1000
\righthyphenmin=1000
\def\tagform@#1{\maketag@@@{[#1]\@@italiccorr}}
\setlength\marginparsep{9\p@} \setlength\marginparpush{8\p@}
\setlength\marginparwidth{5.7pc}

\makeatletter
\setlength\arrayrulewidth{.50\p@}
\setlength\arraycolsep{1.5\p@}
\makeatother

\makeatletter

\def\@makefnmark{\raisebox{0.55ex}{$\!$\FonteAppelNote \@thefnmark}}

\def\@Myfnmark{% RedÈfinit la marque de footnote comme un nombre ordinaire
      % french.sty redÈfinit bÍtement \@makefntext => donc j'utilise une autre macro.
      \mbox{\arabic{footnote}.~}%
}

\renewcommand{\@makefntext}[1]{%
   % Place une footnote en bas de page avec le numÈro \thefootnote
   % \@prevpage sert ‡ dÈtecter les notes de bas de page ‡ cheval sur 2 pages
   \FonteNoteBasPage %NEW
   \noindent\@Myfnmark#1 %
}%

\def\@makesfntext#1{
   % Place une footnote sans numÈro, pour les noms d'auteurs par exemple
   \FonteNoteBasPage %NEW
   \noindent #1 %
}%

\def\@thefnmark{\normalsize\arabic{footnote}}

\makeatother

\newenvironment{abstract}{%
\fontsize{8}{10.5}\sffamily
}
%

\frontmatter

\tableofcontents


\subfile{CHAP_00_FR/Preface_FR}
\subfile{notations_FR}

\mainmatter

\pagenumbering{arabic}

\setcounter{page}{1}
\subfile{CHAP_01_FR/01_Chapitre 1}

\makeatletter
\renewcommand{\@oddhead}{%
            \parbox{\textwidth}{%
                \@DrawPageFrame\hspace*{\fill}\color{chapBlue}\fontfamily{phv}\fontsize{8}{10}\selectfont{D\'etection de changements dans des s\'eries chronologiques d'images SAR polarim\'etrique\hspace{\@ESPage}\thepage\\\vskip-13.5pt\textcolor{chapBlue}\hrule}}}%
\makeatother


\subfile{CHAP_02_FR/02_Chapitre 2}

\makeatletter
\renewcommand{\@oddhead}{%
            \parbox{\textwidth}{%
                \@DrawPageFrame\hspace*{\fill}\color{chapBlue}\fontfamily{phv}\fontsize{8}{10}\selectfont{An Overview of Covariance-based Change Detection Methodologies\hspace{\@ESPage}\thepage\\\vskip-13.5pt\textcolor{chapBlue}\hrule}}}%
\makeatother
\makeatletter
\def\@makechapterhead#1{%
  \vspace*{\@DessusTitreCN}           %
  {\parindent \z@ \iftreatise\centering\else\centering  \fi \normalfont \parskip \z@ %
    \ifnum \c@secnumdepth >\m@ne      %
      \if@mainmatter                  %
%        {\vskip-1pt\fontsize{22}{24}\selectfont\fontfamily\arial\thechapter}\\[\@Intertitre] %
{\fontfamily{phv}\selectfont\color{darkBlue}\fontsize{20}{20}\bfseries\selectfont\thechapter\par}%
       %\textcolor{Blue}\hrule
                  \vskip16pt
      \fi
          \fi                                     %
    \interlinepenalty\@M                    %
    {                                       %
    \if@mainmatter                          %
        \FonteTitreChapitre                 %
    \else                                   %
        \FonteSousTitreChapitre             %
    \fi                                     %
%    \fontsize{18}{20}\selectfont\chtitle{#1}\\[\@DessousTitreCN]}%
    {\fontfamily{phv}\selectfont\fontsize{22}{22}\selectfont{#1}}\\\vspace*{24pt}\@sfootnotetext{1. See \citet{HUSSAIN201391} or \citet[pp. 145--178]{Hecheltjen2014} for overviews.\vspace*{8pt}}} %\vspace*{51.5pt}}%
} %\vskip-91pt%\textcolor{Blue}\hrule
%\vskip11.5mm
}%
\makeatother

\makeatletter
\renewenvironment{authorname}[2]
{%
\begin{tabular}{>{\centering}m{0.98\textwidth}}
{\fontsize{11}{11}\selectfont \bf #1}\\%[3pt]
{\fontsize{9}{9}\selectfont \it #2}
\end{tabular}\vspace*{40pt}\par}
\makeatother

\subfile{CHAP_03_FR/03_Chapitre 3}

\makeatletter
\renewenvironment{authorname}[2]
{%
\begin{tabular}{>{\centering}m{0.9\textwidth}}
{\fontsize{11}{11}\selectfont \bf #1}\\%[3pt]
{\fontsize{9}{9}\selectfont \it #2}
\end{tabular}\vspace*{40pt}\par}

\makeatother

\makeatletter
\def\@makechapterhead#1{% adaptation de book.cls
% Affichage du titre du chapitre numéroté et ne figurant ni dans TOC, ni dans les En-têtes
% de page. Exemples : préface, table des matières (mais avec en-tête)...
  \vspace*{\@DessusTitreCN}           %
  {\parindent \z@ \iftreatise\centering\else\centering  \fi \normalfont \parskip \z@ %
    \ifnum \c@secnumdepth >\m@ne      %
      \if@mainmatter                  %
%        {\vskip-1pt\fontsize{22}{24}\selectfont\fontfamily\arial\thechapter}\\[\@Intertitre] %
{\fontfamily{phv}\selectfont\color{darkBlue}\fontsize{20}{20}\bfseries\selectfont\thechapter\par}%
       %\textcolor{Blue}\hrule
                  \vskip16pt
      \fi
          \fi                                     %
    \interlinepenalty\@M                    %
    {                                       %
    \if@mainmatter                          %
        \FonteTitreChapitre                 %
    \else                                   %
        \FonteSousTitreChapitre             %
    \fi                                     %
%    \fontsize{18}{20}\selectfont\chtitle{#1}\\[\@DessousTitreCN]}%
    {\fontfamily{phv}\selectfont\fontsize{22}{22}\selectfont{#1}}\\\vspace*{24pt}} %\vspace*{51.5pt}}%
} %\vskip-91pt%\textcolor{Blue}\hrule
%\vskip11.5mm
}%
\makeatother
\vfill\eject
\makeatletter
\renewcommand{\@oddhead}{%
            \parbox{\textwidth}{%
                \@DrawPageFrame\hspace*{\fill}\color{chapBlue}\fontfamily{phv}\fontsize{8}{10}\selectfont{Unsupervised Functional Information Clustering in Extreme Environments\hspace{\@ESPage}\thepage\\\vskip-13.5pt\textcolor{chapBlue}\hrule}}}%
\makeatother
\subfile{CHAP_04_FR/04_Chapitre 4} %

\makeatletter
\renewcommand{\@oddhead}{%
            \parbox{\textwidth}{%
                \@DrawPageFrame\hspace*{\fill}\color{chapBlue}\fontfamily{phv}\fontsize{8}{10}\selectfont{Thresholds and Distances with Sentinel-1 Image Time Series\hspace{\@ESPage}\thepage\\\vskip-13.5pt\textcolor{chapBlue}\hrule}}}%
\makeatother
\vfill\eject
\subfile{CHAP_05_FR/05_Chapitre 5} %

\makeatletter
\renewcommand{\@oddhead}{%
            \parbox{\textwidth}{%
                \@DrawPageFrame\hspace*{\fill}\color{chapBlue}\fontfamily{phv}\fontsize{8}{10}\selectfont{Fractional Field Image Time Series Modeling and Application\hspace{\@ESPage}\thepage\\\vskip-13.5pt\textcolor{chapBlue}\hrule}}}%
\makeatother
\vfill\eject
\subfile{CHAP_06_FR/06_Chapitre 6}%
\vfill\eject
\makeatletter
\renewcommand{\@oddhead}{%
            \parbox{\textwidth}{%
                \@DrawPageFrame\hspace*{\fill}\color{chapBlue}\fontfamily{phv}\fontsize{8}{10}\selectfont{Graph of Characteristic Points for Texture Tracking\hspace{\@ESPage}\thepage\\\vskip-13.5pt\textcolor{chapBlue}\hrule}}}%
\makeatother
\subfile{CHAP_07_FR/07_Chapitre 7}

\makeatletter
\renewcommand{\@oddhead}{%
            \parbox{\textwidth}{%
                \@DrawPageFrame\hspace*{\fill}\color{chapBlue}\fontfamily{phv}\fontsize{8}{10}\selectfont{Multitemporal Analysis of Sentinel-1/2 Images for Land Use Monitoring\hspace{\@ESPage}\thepage\\\vskip-13.5pt\textcolor{chapBlue}\hrule}}}%
\makeatother
\vfill\eject
\subfile{CHAP_08_FR/08_Chapitre 8}

\makeatletter
\renewcommand{\@oddhead}{%
            \parbox{\textwidth}{%
                \@DrawPageFrame\hspace*{\fill}\color{chapBlue}\fontfamily{phv}\fontsize{8}{10}\selectfont{Statistical Difference Models for Change Detection in Multispectral Images\hspace{\@ESPage}\thepage\\\vskip-13.5pt\textcolor{chapBlue}\hrule}}}%
\makeatother
\vfill\eject
\subfile{CHAP_09_FR/09_Chapitre 9}

\makeatletter
\renewcommand{\@oddhead}{%
            \parbox{\textwidth}{%
                \@DrawPageFrame\hspace*{\fill}\color{chapBlue}\fontfamily{phv}\fontsize{8}{10}\selectfont{\VAR@ChapitreAbrege\hspace{\@ESPage}\thepage\\\vskip-13.5pt\textcolor{chapBlue}\hrule}}}%
                \makeatother
\vfill\eject
\subfile{listofauthors_FR}


%\backmatter
%\printindex

\begin{theindex}
\mkidxletter{A, B}
  \item anomaly detection, 110, 113, 116, 119
  \item Area Under Curve (AUC), 7, 22
  \item AutoRegressive (AR), 161
  \subitem Fractionally Integrated Moving Average (ARFIMA),
		161
  \subitem Moving Average (ARMA), 161
  \item Bayesian decision, 233
  \item Bi-Date Features -- Multi-Date Divergence Matrices (BDF-MDDM),
		113, 116
\mkidxletter{C}
  \item centroid, 115, 116
  \item Change
  \subitem Detection (CD), 2, 35, 53, 73, 84, 168, 175, 181, 223, 263
  \subitem map, 204, 214
  \subitem Measure (CM), 178, 188
  \subitem Vector Analysis (CVA), 5, 6, 230, 240
  \item clustering, 110, 113, 119
  \item Commission Errors (CE), 7, 22, 27
  \item Complex
  \subitem Compound Gaussian (CCG), 84
  \subitem Elliptical Symmetric (CES), 83
  \item Compressed Change Vector Analysis (C$^{2}$VA), 6, 9
  \item Constant False Alarm Rate (CFAR), 85
  \item Convolutional Neural Network (CNN/convnet), 109, 110, 112, 113, 119
  \item correlation matrix, 139
  \item Cumulant-based Kullback--Leibler Divergence (CKLD), 168, 179,
		184
  \item cyclone, 145, 147, 148
  \subitem eye/eyewall, 146
\mkidxletter{D, E}
  \item detection, 146, 157
  \item Difference Image (DI), 168
  \item Digital Elevation Model (DEM), 129
  \item Discrete Wavelet Packet Transform (DWPT), 149, 153
  \item Discrete Wavelet Transform (DWT), 149, 153
  \item Earth Observation (EO), 223
  \item Empirical Mode Decomposition (EMD), 169
  \item Expectation--Maximization (EM) algorithm, 231, 225, 235
  \mkidxletter{F, G}
  \item False Alarm (FA), 38, 180
    \item False Negative Rate (FNR)/miss rate, 38, 180
  \item False Negatives (FN), 38, 180
  \item False Positive Rate (FPR), 38
  \item False Positives (FP), 180
  \item fractal intensity parameters, 155, 156
  \item Fuzzy C-Means (FCM), 18
  \item Gamma-MAP, 210
  \item Gaussian
    \subitem -based Kullback--Leibler Divergence (GKLD), 179, 184
    \subitem distribution, 77
    \subitem mixture, 257
  \item Generalized Likelihood Ratio Test (GLRT), 85, 97
  \item Geostationary Operational Environmental Satellite (GOES), 147
  \item Good Detection (GD), 180
  \item Google Earth Engine (GEE), 202, 208
  \item Ground Range Detected (GRD), 129, 202
\mkidxletter{H, I}
  \item Harris corner points, 174
  \item Hausdorff distance, 133
  \item High Resolution (HR), 2
  \item hyperspectral image, 9, 11
  \item Image Time Series (ITS), 76, 145
  \item Independent Component Analysis (ICA), 5
  \item Interferometric Synthetic Aperture Radar (InSAR), 202
  \item Intrinsic Mode Function (IMF), 170
  \item Iteratively Reweighted Multivariate Alteration Detection (IR-MAD),
		22, 27, 229
\mkidxletter{K, L}
  \item K-means, 182, 184
  \item Kappa coefficient, 6, 22, 27
  \item Kittler--Illingworth (KI) thresholding, 182, 184
  \item Kullback--Leibler (KL), 87
  \subitem Divergence (KLD), 168, 177, 211--213
    \item Landsat, 224, 253, 254
  \item likelihood ratio test, 39, 45, 47, 49--52, 54, 55
  \item localization, 146, 158
  \item Log-Ratio Detector (LRD), 168, 177, 179, 184
  \item Low-Rank Compound Gaussian (LRCG), 97
\mkidxletter{M}
  \item Markov Random Field (MRF), 233
  \item matrix norm, 133
  \item Maximum
  \subitem \textit{a posteriori} (MAP), 228
  \subitem Likelihood (ML), 234
  \subsubitem Estimator (MLE), 77, 98, 99
  \item Mean-Ratio Detector (MRD), 168, 179, 184
  \item Method of Moments (MOM), 234
  \item Missed Detection (MD), 181
  \item model identification, 148
  \item Modified Normalized Difference Water Index (MNDWI), 5
  \item Moving Average (MA), 161
    \item Multi-Date Divergence Matrices (MDDM), 110, 113, 114
  \item Multi-Date Features-Multi -- Date Divergence Matrices (MDF-MDDM),
		116, 119, 120
  \item multiclass Rician mixture, 260
      \item Multiscale Morphological Compressed Change Vector Analysis (M$^{2}$C$^{2}$VA),
		11, 22, 27
  \item multispectral, 203, 253, 254
 \subitem image, 5, 7, 8, 11
  \subitem instrument (MSI), 254
    \item Multivariate Alteration Detection (MAD), 229
\mkidxletter{N, O}
  \item Non-Local Mean (NLM), 179
  \item Normalized
  \subitem Cross Correlation Ratio (NCCR), 133
  \subitem Difference Vegetation Index (NDVI), 5, 216
  \subitem Difference Water Index (NDWI), 216
  \item Omission Errors (OE), 7, 22, 27
  \item Operational Land Imager (OLI), 254
  \item Otsu thresholding, 182, 184
  \item Overall Accuracy (OA), 7, 22, 27, 181
\mkidxletter{P}
    \item parameter estimation, 153, 154
  \item Pearson-based Kullback--Leibler Divergence (PKLD), 168
  \item Polarimetric Synthetic Aperture Radar (PolSAR), 35, 119, 123
  \item Power Spectral Density (PSD), 150, 153
  \item prediction, 146, 158, 159
  \item Principal Components Analysis (PCA), 5
  \item Projection-based Jeffrey Detector (PJD), 169
\item $P_{\textrm{value}}$, $p$-value, 38
\mkidxletter{R}
  \item RADARSAT-1, 179
  \item RADARSAT-2, 180
  \item Rayleigh-Rice mixture, 257, 260
  \item Receiver Operating Characteristic (ROC), 7, 22, 91, 180
  \item Region of Interest (ROI), 180, 204
  \item Reliability Index (RI), 194
  \item Rician mixture, 257
\mkidxletter{S}
  \item Sample Covariance Matrix (SCM), 77
  \item SAR pre-processing
  \subitem border noise removal, 209
  \subitem radiometric calibration, 209
  \subitem terrain correction, orthorectification,
		209
  \subitem (thermal noise removal), 209
  \item Scale-Invariant Feature Transform (SIFT), 174
      \item Sentinel-1, 118, 119, 123, 127, 203, 208
  \item Sentinel-2, 203, 208
  \item Sentinel Application Platform (SNAP), 202
  \item Sequential Spectral Change Vector Analysis (S$^{2}$CVA), 9, 22,
		27
  \item significance level, 38
  \item Simple Linear Iterative Clustering (SLIC), 15
  \item Spectral
  \subitem analysis, 148, 150, 169, 176
  \subitem Angle Distance (SAD), 9
  \subitem Change Vector (SCV), 6, 9
  \item Speeded Up Robust Features (SURF), 174
  \item Stationary
  \subitem Wavelet Packet Transform (SWPT), 149, 153
  \subitem Wavelet Transform (SWT), 149, 153
  \item SuperPixel-level Compressed Change Vector Analysis (SPC$^{2}$VA),
		15, 22, 27
  \item Synthetic Aperture Radar (SAR), 35, 73, 110, 127, 167, 202, 203

\mkidxletter{T}
  \item Terell gradient, 85
  \item TerraSAR-X (TSX), 194
  \item Total Errors (TE), 22, 27, 181
  \item tracking, 158
  \item True
  \subitem Negatives (TN), 180
  \subitem Positives (TP), 180
\mkidxletter{U, V, W}
  \item UAVSAR, 76, 79--81
  \item Very High Resolution (VHR), 2, 8, 229
  \item Wald statistics, 85
  \item wavelet transform, 110, 113, 119, 148--150, 168, 169, 176
  \item Wishart distribution, 35

\end{theindex}

\pagebreak
\
\thispagestyle{empty}

\end{document}
